\documentclass[12pt]{article}
\usepackage{amsmath, amssymb}
\usepackage[brazil]{babel}
\usepackage[utf8]{inputenc}

\title{Propriedades de Números Pares e Ímpares}
\author{}
\date{}

\begin{document}

\maketitle

\section*{Definições de Paridade}

\begin{itemize}
    \item Um número inteiro \( n \) é \textbf{par} se \( \dfrac{n}{2} \in \mathbb{Z} \). Isso equivale a dizer que existe \( k \in \mathbb{Z} \) tal que \( n = 2k \).
    \item Um número inteiro \( n \) é \textbf{ímpar} se \( \dfrac{n - 1}{2} \in \mathbb{Z} \). Isso equivale a dizer que existe \( k \in \mathbb{Z} \) tal que \( n = 2k + 1 \).
\end{itemize}

\section*{1. Se \( n \) é par, então \( n^2 \) é par}

\textbf{Prova:}  
Se \( n \) é par, então \( n = 2k \) para algum \( k \in \mathbb{Z} \).  
Logo,
\[
n^2 = (2k)^2 = 4k^2 = 2(2k^2)
\]
Como \( 2k^2 \in \mathbb{Z} \), segue que \( \dfrac{n^2}{2} \in \mathbb{Z} \).  
Portanto, \( n^2 \) é par.

\hfill\(\blacksquare\)

\section*{2. Se \( n \) é ímpar, então \( n^2 \) é ímpar}

\textbf{Prova:}  
Se \( n \) é ímpar, então \( n = 2k + 1 \) para algum \( k \in \mathbb{Z} \).  
Logo,
\[
n^2 = (2k + 1)^2 = 4k^2 + 4k + 1 = 2(2k^2 + 2k) + 1
\]
Como \( 2k^2 + 2k \in \mathbb{Z} \), \( n^2 \) tem a forma \( 2m + 1 \), logo é ímpar.

\hfill\(\blacksquare\)

\section*{3. Se \( n^2 \) é ímpar, então \( n \) é ímpar}

\textbf{Prova por contrarrecíproca:}  
Se \( n \) fosse par, então \( n = 2k \Rightarrow n^2 = 4k^2 = 2(2k^2) \), que é par.  
Logo, se \( n^2 \) é ímpar, \( n \) não pode ser par, ou seja, \( n \) é ímpar.

\hfill\(\blacksquare\)

\section*{4. Se \( n \) é par, então \( n \) não é ímpar}

\textbf{Prova por contradição:}  
Suponha que \( n \) seja par e ímpar:  
\[
n = 2k \quad \text{e} \quad n = 2m + 1
\Rightarrow 2k = 2m + 1 \Rightarrow k - m = \frac{1}{2}
\]
Isso contradiz o fato de que \( k, m \in \mathbb{Z} \).  
Logo, \( n \) não pode ser par e ímpar ao mesmo tempo.

\hfill\(\blacksquare\)

\section*{5. Todo número inteiro é par ou ímpar, mas não ambos}

\textbf{Prova:}  
Seja \( n \in \mathbb{Z} \).  
Se \( \dfrac{n}{2} \in \mathbb{Z} \), então \( n \) é par.  
Se não, então \( n = 2k + 1 \) para algum \( k \in \mathbb{Z} \), ou seja, \( n \) é ímpar.

Já mostramos que um número não pode ser par e ímpar ao mesmo tempo.  
Portanto, todo número inteiro é ou par ou ímpar, e nunca os dois.

\hfill\(\blacksquare\)

\end{document}
